% sprawozdanie_projekt_udostępnianie_ekranu.tex
% Projekt indywidualny - system do udostępniania ekranu 
% Jordan Parviainen 2023

\documentclass[a4paper,11pt]{article}

% Packages
\usepackage[utf8]{inputenc}  		% Kodowanie pliku
\usepackage[T1]{fontenc}
\usepackage{graphicx}			% Grafika
\usepackage[inkscapeformat=pdf]{svg}	% .svg tylko w postaci wektorowej - nie png! .pdf/.eps/.ps/.png/.pdf_tex

% Settings
\hoffset=-3.0cm                         % Mniejszy lewy margines
\textwidth=18cm                         % szerzej
\evensidemargin=0pt

\voffset=-3cm                           % Mniejszy górny margines
\textheight=27cm                        % szerzej wzdłuż

\setlength{\parindent}{0pt}             % Paragraf od początku linii
\setlength{\parskip}{\medskipamount}    % Odstęp pomiędzy paragrafami
\raggedbottom                           % bez rozciągania strony
\frenchspacing                          % Bez spacji na końcu zdania.
%\graphicspath{ {c1} }


% Dodatkowe komendy
\newcommand\BS{\char`\\}                % \BS == back-slash
\newcommand\TY{\raise.17ex\hbox{$\scriptstyle\mathtt{\sim}$}}   % \TY == większa tylda w \tt

\title{%
    Sprawozdanie z projektu indywidualnego \\
    \ \\
    \large{Temat: Wdrożenie prostego systemu do udostępniania ekranu komputera w sieci lokalnej}}
\author{Jordan Parviainen}
\begin{document}
    \maketitle
    \newpage
    \pagenumbering{arabic}
    \section{Postawienie problemu i zadania}
        \subsection{Potrzeba}
        W dobie powszechnej informatyzacji i pracy zdalnej, coraz więcej rzeczy jest wykonywanych przy pomocy komputera. Rośnie wraz z tym potrzeba sprawnej współpracy pomiędzy użytkownikami.  
        Bardzo przydatna jest możliwość pokazania jak wykonać daną czynność poprzez udostępnienie obrazu swojego komputera, czy to na żywo, czy to nagrywając materiał wideo. 
        Z takimi potrzebami spotyka się także na uczelni, np. podczas zajęć laboratoryjnych z wykorzystaniem komputerów, gdzie często prowadzący pokazuje jak wykonać daną czynność, a studenci obserwują.
        Do tego celu wykorzystuje się często rzutniki, jednak nie jest to zawsze rozwiązanie wygodne czy optymalne. Bywają problemy z widocznością obrazu, a także nie każda sala jest do tego przystosowana.  
        \subsection{Pomysł na rozwiązanie}
        Aby znaleźć alternatywny sposób na rozwiązanie tego problemu, zamiast patrzeć w stronę sprzętu, można zwrócić uwagę na oprogramowanie. Czy nie byłoby łatwiej, gdyby studenci mogli oglądać ekran prowadzącego na swoich komputerach?
        Istnieją oczywiście już tego typu rozwiązania, powszechnie stosowane w pracy zdalnej, np. MSTeams, Zoom. Jednakże są to rozwiązania zbyt rozbudowane, a także nie zawsze dostępne. Nie są one dobrym środkiem 
        w przypadku np. zajęć laboratoryjnych, gdzie trzeba na szybko jedynie pokazać ekran. Perspektywa instalowania ciężkiego oprogramowanie i tworzenia wideokonferencji, dodawania użytkowników etc. jest odstraszająca.
        Dlatego też powstał pomysł stworzenia prostego systemu udostępniania obrazu ekranu komputera, który idealnie nie wymagałby instalacji oprogramowania klienckiego, ani długiej konfiguracji. 
    \section{Przegląd istniejących rozwiązań}
        \subsection{Programy do wideokonferencji}
        Jak już wspomniałem, istnieje wiele programów do wideokonferencji, które w ramach swojej funkcjonalności oferują udostępnianie ekranu. Są to np. MSTeams, Zoom, Google Meet, Skype, etc.
        Jednak w zastosowaniu do udostępniania obrazu ad-hoc np. w sieci lokalnej w laboratorium mają liczne wady, takie jak:
        \begin{itemize}
            \item Konieczność logowania/zakładania konta
            \item Konieczność instalowania oprogramowania klienckiego, często niemałego w rozmiarach. Co prawda istnieją wersje przeglądarkowe, ale bywają okrojone w funkcjonalności i nie działają optymalnie. 
            \item Problemy z jakością/obciążeniem sieci. W przypadku, gdy trzeba udostępnić obraz wielu użytkownikom w jednej sali, wysoce nieoptymalnym jest korzystanie z oprogramowania, którego strumieniowanie
            odbywa się za pośrednictwem zewnętrznego serwera.   
        \end{itemize}     
        \subsection{Programy typu zdalny pulpit}
        Dostępne są na rynku programy o funkcjonalności pozwalającej na kontrolę nad zdalnym komputerem. 
        Są to m.in. takie programy jak: TightVNC, TeamViewer, Windows Remote Desktop, CoScreen, Anydesk. 
        Rozwiązania te dzielą wady wymienione, powyżej, a ponadto potrafią być zbyt skomplikowane do użytku, jeśli potrzebne jest tylko udostępnianie ekranu. 
        Ponadto często są to rozwiązania komercyjne, o zamkniętym kodzie źródłowym i płatnej licencji. 
        W laboratorium Sieci Komputerowych użyte było rozwiązanie oparte na skrypcie uruchamiającym i koordynującym połączenie za pomocą TightVNC. 
        Niestety, oznaczało to konieczność instalacji TightVNC na każdym komputerze, a samo udostępnianie np. okna terminala obwarowane było ograniczeniami i koniecznością ręcznego ustawiania np. rozdzielczości. 
    \section{Opis rozwiązania}
        \subsection{Pomysł}
        Swoją nadzieję położyłem w nowoczesnej przeglądarkowej technologii WebRTC. Pozwala ona na strumieniowanie obrazu i dźwięku przez przeglądarkę internetową w topologii punkt-do-punktu. 
        Jest to kompleksowe rozwiązanie obejmujące kompresję, kodowanie, szyfrowanie strumienia i wykorzystanie akceleracji sprzętowej.
        Ponadto technologia zawiera też mechanizmy zapewniające stałe bezpośrednie połączenie pomiędzy użytkownikami, nawet w skomplikowanych konfiguracjach sieciowych z NAT-ami.
        Cała ta moc jest udostępniona przez JavaScript-owe API w każdej popularnej przeglądarce internetowej (silniki Chromium i Gecko), oprócz tych opartych na silniku WebKit w produktach Apple. 
        \subsection{Poszukiwanie gotowych rozwiązań}
        Możnaby pokusić się o samodzielne napisanie rozwiązania, jednak wolałem przeszukać zasoby gotowych rozwiązań o otwartym kodzie źródłowym,
        czy przypadkiem jakaś część tego, co jest potrzebne nie została już przez kogoś wykonana.
        Rozpocząłem swoje poszukiwania na najpopularniejszej stronie do udostępniania kodu - \emph{github.com}. Dosyć szybko natrafiłem, na to, czego szukałem.
        \subsection{Znalezione oprogramowanie}
        Znalazłem publicznie dostępny kod aplikacji internetowej wykorzystującej technologię WebRTC do udostępniania ekranu z poziomu przeglądarki. \\
        Repozytorium znajduje się pod adresem github.com/adamyordan/laplace. \\
        Jest to prosty serwer HTTP napisany w języku Go. 
        Serwuje on pliki części frontendowej i komunikuje się z nią za pomocą protokołu WebSocket w celu negocjacji połączenia WebRTC (signaling). 
        Część kliencka (przeglądarkowa) aplikacji napisana jest w czystym Javascript. Cała aplikacja ma niewiele funkcjonalności, ale za to kod jest zwięzły i całość ma niewielki rozmiar.
        \subsection{Działanie aplikacji}
        Działanie aplikacji polega na tym, że gdy ktoś chce udostępnić innym swój ekran, wybiera odpowiednią opcję i tworzy tzw. "pokój", 
        który ma przypisany identyfikator i do którego mogą dołączyć inni używając tego identyfikatora. 
        Zgodnie ze specyfikacją WebRTC, serwer nie pośredniczy w strumieniowaniu obrazu, a jedynie w nawiązaniu połączenia i jego utrzymaniu między użytkownikami.   
        Maszyna osoby udostępniającej ekran wysyła pakiety transmisji do każdej maszyny oglądającego. 
        Poniżej znajduje się schemat przedstawiający architekturę i działanie aplikacji: 
        \begin{figure}[!ht]
        \centering
        \def\svgwidth{\columnwidth}
        \includesvg[width=0.8\columnwidth]{schemat_dzialania}
        \caption{Schemat sieci wewnętrznej}
        \label{rys1:label}
        \end{figure}
        \subsection{Więcej o WebRTC}
        Nawiązywanie połączenia punkt-do-punktu zaczyna się od zgłoszenia się maszyny chcącej nawiązać połączenie do serwera sygnalizacyjnego(ang.\emph{signaling server}).
        Następnie, serwer sygnalizacyjny ustala parametry połączenia nowej maszyny do pozostałych, korzystając z technologii ICE - (ang.\emph{Interactive Connectivity Establishment}).
        Jest to protokół służący do znajdowania sposobu, w jaki komputery mogą być ze sobą połączone jak najbardziej bezpośrednio, jak to jest możliwe. 
        Wykrywa możliwości obchodzenia m.in. NAT-ów i firewall-i. 
        W tym procesie znajdowani są tzw. kandydaci ścieżek i adresów połączeń uszeregowanych według skuteczności i informacja ta jest przesyłana do klientów, 
        którzy następnie nawiązują według tych informacji połączenia bezpośrednio z innymi klientami.  
    \section{Wdrożenie rozwiązania} 
        \subsection{Środowisko wdrożeniowe i cel}
        Celem projektu jest wdrożenie opisanego przeze mnie rozwiązania do laboratorium Sieci Komputerowych w gmachu Elektrycznym Politechniki Warszawskiej. 
        Nie jest to jedyny cel, idealnie byłoby przygotować prostą ścieżkę do zastosowania podobnego rozwiązania w innych, podobnych środowiskach.    
        Laboratorium Sieci Komputerowych składa się z kilkunastu maszyn połączonych w lokalną sieć o dużej przepustowości zarówno wewnętrznej, jak i do internetu.
        Na maszynach tych standardowo używany jest Arch Linux podnoszony z sieci, więc instalacja i konfiguracja była robiona z myślą o tym środowisku.  
        \subsection{Co było do zrealizowania}
        W ramach wdrożenia zrealizowałem następujące rzeczy: 
        \begin{itemize}
            \item Dostosowałem oprogramowanie do warunków sieci lokalnej i przygotowałem jego repozytorium.
            \item Napisałem skrypty instalacyjne pod system operacyjny Arch Linux.
            \item Napisałem skrypty usprawniające pracę z oprogramowaniem do udostęniania ekranu.
        \end{itemize}
        \subsection{Modyfikacja oprogramowania}
        \begin{itemize}
            \item Dodałem punkt końcowy do interfejsu REST serwera, zwracający identyfikator ostatnio utworzonego "pokoju", co umożliwia automatyzację otwierania ekranu udostępnianego przez prowadzącego w laboratorium.  
            \item Dodałem możliwość przekazania w fazie kompilacji zmiennej z lokalizacją plików statycznych serwera HTTP.
            \item Ustawiłem najwyższe ustawienie jakości jako domyślne. 
        \end{itemize}
        Stworzyłem repozytorium na GitHub'ie i nadałem projektowi anglosaską nazwę "projector".
        \subsection{Skrypty instalacyjne}
        Na Arch Linuxie pakiety są budowane i instalowane za pomocą programu \texttt{makepkg}, według procedury opisanej w ustandaryzowanym pliku \texttt{PKGBUILD} pakietu. 
        Oto ten skrypt dla programu "projector":
        \lstinputlisting{PKGBUILD}
        Pełnej automatyzacji budowania i instalacji dopełnia skrypt pomocniczy:
        \lstinputlisting{proj}
        Pobiera on plik \texttt{PKGBUILD} z repozytorium i uruchamia makepkg z odpowiednimi flagami. 
    \section{Testy rozwiązania}
\end{document}